% Options for packages loaded elsewhere
\PassOptionsToPackage{unicode}{hyperref}
\PassOptionsToPackage{hyphens}{url}
%
\documentclass[
]{article}
\usepackage{amsmath,amssymb}
\usepackage{iftex}
\ifPDFTeX
  \usepackage[T1]{fontenc}
  \usepackage[utf8]{inputenc}
  \usepackage{textcomp} % provide euro and other symbols
\else % if luatex or xetex
  \usepackage{unicode-math} % this also loads fontspec
  \defaultfontfeatures{Scale=MatchLowercase}
  \defaultfontfeatures[\rmfamily]{Ligatures=TeX,Scale=1}
\fi
\usepackage{lmodern}
\ifPDFTeX\else
  % xetex/luatex font selection
\fi
% Use upquote if available, for straight quotes in verbatim environments
\IfFileExists{upquote.sty}{\usepackage{upquote}}{}
\IfFileExists{microtype.sty}{% use microtype if available
  \usepackage[]{microtype}
  \UseMicrotypeSet[protrusion]{basicmath} % disable protrusion for tt fonts
}{}
\makeatletter
\@ifundefined{KOMAClassName}{% if non-KOMA class
  \IfFileExists{parskip.sty}{%
    \usepackage{parskip}
  }{% else
    \setlength{\parindent}{0pt}
    \setlength{\parskip}{6pt plus 2pt minus 1pt}}
}{% if KOMA class
  \KOMAoptions{parskip=half}}
\makeatother
\usepackage{xcolor}
\usepackage[margin=1in]{geometry}
\usepackage{color}
\usepackage{fancyvrb}
\newcommand{\VerbBar}{|}
\newcommand{\VERB}{\Verb[commandchars=\\\{\}]}
\DefineVerbatimEnvironment{Highlighting}{Verbatim}{commandchars=\\\{\}}
% Add ',fontsize=\small' for more characters per line
\usepackage{framed}
\definecolor{shadecolor}{RGB}{248,248,248}
\newenvironment{Shaded}{\begin{snugshade}}{\end{snugshade}}
\newcommand{\AlertTok}[1]{\textcolor[rgb]{0.94,0.16,0.16}{#1}}
\newcommand{\AnnotationTok}[1]{\textcolor[rgb]{0.56,0.35,0.01}{\textbf{\textit{#1}}}}
\newcommand{\AttributeTok}[1]{\textcolor[rgb]{0.13,0.29,0.53}{#1}}
\newcommand{\BaseNTok}[1]{\textcolor[rgb]{0.00,0.00,0.81}{#1}}
\newcommand{\BuiltInTok}[1]{#1}
\newcommand{\CharTok}[1]{\textcolor[rgb]{0.31,0.60,0.02}{#1}}
\newcommand{\CommentTok}[1]{\textcolor[rgb]{0.56,0.35,0.01}{\textit{#1}}}
\newcommand{\CommentVarTok}[1]{\textcolor[rgb]{0.56,0.35,0.01}{\textbf{\textit{#1}}}}
\newcommand{\ConstantTok}[1]{\textcolor[rgb]{0.56,0.35,0.01}{#1}}
\newcommand{\ControlFlowTok}[1]{\textcolor[rgb]{0.13,0.29,0.53}{\textbf{#1}}}
\newcommand{\DataTypeTok}[1]{\textcolor[rgb]{0.13,0.29,0.53}{#1}}
\newcommand{\DecValTok}[1]{\textcolor[rgb]{0.00,0.00,0.81}{#1}}
\newcommand{\DocumentationTok}[1]{\textcolor[rgb]{0.56,0.35,0.01}{\textbf{\textit{#1}}}}
\newcommand{\ErrorTok}[1]{\textcolor[rgb]{0.64,0.00,0.00}{\textbf{#1}}}
\newcommand{\ExtensionTok}[1]{#1}
\newcommand{\FloatTok}[1]{\textcolor[rgb]{0.00,0.00,0.81}{#1}}
\newcommand{\FunctionTok}[1]{\textcolor[rgb]{0.13,0.29,0.53}{\textbf{#1}}}
\newcommand{\ImportTok}[1]{#1}
\newcommand{\InformationTok}[1]{\textcolor[rgb]{0.56,0.35,0.01}{\textbf{\textit{#1}}}}
\newcommand{\KeywordTok}[1]{\textcolor[rgb]{0.13,0.29,0.53}{\textbf{#1}}}
\newcommand{\NormalTok}[1]{#1}
\newcommand{\OperatorTok}[1]{\textcolor[rgb]{0.81,0.36,0.00}{\textbf{#1}}}
\newcommand{\OtherTok}[1]{\textcolor[rgb]{0.56,0.35,0.01}{#1}}
\newcommand{\PreprocessorTok}[1]{\textcolor[rgb]{0.56,0.35,0.01}{\textit{#1}}}
\newcommand{\RegionMarkerTok}[1]{#1}
\newcommand{\SpecialCharTok}[1]{\textcolor[rgb]{0.81,0.36,0.00}{\textbf{#1}}}
\newcommand{\SpecialStringTok}[1]{\textcolor[rgb]{0.31,0.60,0.02}{#1}}
\newcommand{\StringTok}[1]{\textcolor[rgb]{0.31,0.60,0.02}{#1}}
\newcommand{\VariableTok}[1]{\textcolor[rgb]{0.00,0.00,0.00}{#1}}
\newcommand{\VerbatimStringTok}[1]{\textcolor[rgb]{0.31,0.60,0.02}{#1}}
\newcommand{\WarningTok}[1]{\textcolor[rgb]{0.56,0.35,0.01}{\textbf{\textit{#1}}}}
\usepackage{graphicx}
\makeatletter
\def\maxwidth{\ifdim\Gin@nat@width>\linewidth\linewidth\else\Gin@nat@width\fi}
\def\maxheight{\ifdim\Gin@nat@height>\textheight\textheight\else\Gin@nat@height\fi}
\makeatother
% Scale images if necessary, so that they will not overflow the page
% margins by default, and it is still possible to overwrite the defaults
% using explicit options in \includegraphics[width, height, ...]{}
\setkeys{Gin}{width=\maxwidth,height=\maxheight,keepaspectratio}
% Set default figure placement to htbp
\makeatletter
\def\fps@figure{htbp}
\makeatother
\setlength{\emergencystretch}{3em} % prevent overfull lines
\providecommand{\tightlist}{%
  \setlength{\itemsep}{0pt}\setlength{\parskip}{0pt}}
\setcounter{secnumdepth}{-\maxdimen} % remove section numbering
\ifLuaTeX
  \usepackage{selnolig}  % disable illegal ligatures
\fi
\IfFileExists{bookmark.sty}{\usepackage{bookmark}}{\usepackage{hyperref}}
\IfFileExists{xurl.sty}{\usepackage{xurl}}{} % add URL line breaks if available
\urlstyle{same}
\hypersetup{
  pdftitle={TP1 - Inferencia Estadística y Reconocimiento de Patrones},
  hidelinks,
  pdfcreator={LaTeX via pandoc}}

\title{TP1 - Inferencia Estadística y Reconocimiento de Patrones}
\author{}
\date{\vspace{-2.5em}}

\begin{document}
\maketitle

\hypertarget{parte-de-regresiuxf3n-en-r}{%
\section{Parte de regresión en R}\label{parte-de-regresiuxf3n-en-r}}

\hypertarget{carga-de-datos-y-exploraciuxf3n}{%
\subsection{Carga de datos y
exploración}\label{carga-de-datos-y-exploraciuxf3n}}

\begin{Shaded}
\begin{Highlighting}[]
\CommentTok{\# Borrar variables previas  rm(list = ls())}
\CommentTok{\# Cargar el dataset}
\NormalTok{vinos }\OtherTok{\textless{}{-}} \FunctionTok{read.csv2}\NormalTok{(}\StringTok{"winequality{-}red.csv"}\NormalTok{, }\AttributeTok{dec =} \StringTok{"."}\NormalTok{)}
\CommentTok{\# Nombres y estructura}
\FunctionTok{names}\NormalTok{(vinos)}
\end{Highlighting}
\end{Shaded}

\begin{verbatim}
##  [1] "fixed.acidity"        "volatile.acidity"     "citric.acid"         
##  [4] "residual.sugar"       "chlorides"            "free.sulfur.dioxide" 
##  [7] "total.sulfur.dioxide" "density"              "pH"                  
## [10] "sulphates"            "alcohol"              "quality"
\end{verbatim}

\begin{Shaded}
\begin{Highlighting}[]
\FunctionTok{head}\NormalTok{(vinos)}
\end{Highlighting}
\end{Shaded}

\begin{verbatim}
##   fixed.acidity volatile.acidity citric.acid residual.sugar chlorides
## 1           7.4             0.70        0.00            1.9     0.076
## 2           7.8             0.88        0.00            2.6     0.098
## 3           7.8             0.76        0.04            2.3     0.092
## 4          11.2             0.28        0.56            1.9     0.075
## 5           7.4             0.70        0.00            1.9     0.076
## 6           7.4             0.66        0.00            1.8     0.075
##   free.sulfur.dioxide total.sulfur.dioxide density   pH sulphates alcohol
## 1                  11                   34  0.9978 3.51      0.56     9.4
## 2                  25                   67  0.9968 3.20      0.68     9.8
## 3                  15                   54  0.9970 3.26      0.65     9.8
## 4                  17                   60  0.9980 3.16      0.58     9.8
## 5                  11                   34  0.9978 3.51      0.56     9.4
## 6                  13                   40  0.9978 3.51      0.56     9.4
##   quality
## 1       5
## 2       5
## 3       5
## 4       6
## 5       5
## 6       5
\end{verbatim}

\begin{Shaded}
\begin{Highlighting}[]
\FunctionTok{summary}\NormalTok{(vinos)}
\end{Highlighting}
\end{Shaded}

\begin{verbatim}
##  fixed.acidity   volatile.acidity  citric.acid    residual.sugar  
##  Min.   : 4.60   Min.   :0.1200   Min.   :0.000   Min.   : 0.900  
##  1st Qu.: 7.10   1st Qu.:0.3900   1st Qu.:0.090   1st Qu.: 1.900  
##  Median : 7.90   Median :0.5200   Median :0.260   Median : 2.200  
##  Mean   : 8.32   Mean   :0.5278   Mean   :0.271   Mean   : 2.539  
##  3rd Qu.: 9.20   3rd Qu.:0.6400   3rd Qu.:0.420   3rd Qu.: 2.600  
##  Max.   :15.90   Max.   :1.5800   Max.   :1.000   Max.   :15.500  
##    chlorides       free.sulfur.dioxide total.sulfur.dioxide    density      
##  Min.   :0.01200   Min.   : 1.00       Min.   :  6.00       Min.   :0.9901  
##  1st Qu.:0.07000   1st Qu.: 7.00       1st Qu.: 22.00       1st Qu.:0.9956  
##  Median :0.07900   Median :14.00       Median : 38.00       Median :0.9968  
##  Mean   :0.08747   Mean   :15.87       Mean   : 46.47       Mean   :0.9967  
##  3rd Qu.:0.09000   3rd Qu.:21.00       3rd Qu.: 62.00       3rd Qu.:0.9978  
##  Max.   :0.61100   Max.   :72.00       Max.   :289.00       Max.   :1.0037  
##        pH          sulphates         alcohol         quality     
##  Min.   :2.740   Min.   :0.3300   Min.   : 8.40   Min.   :3.000  
##  1st Qu.:3.210   1st Qu.:0.5500   1st Qu.: 9.50   1st Qu.:5.000  
##  Median :3.310   Median :0.6200   Median :10.20   Median :6.000  
##  Mean   :3.311   Mean   :0.6581   Mean   :10.42   Mean   :5.636  
##  3rd Qu.:3.400   3rd Qu.:0.7300   3rd Qu.:11.10   3rd Qu.:6.000  
##  Max.   :4.010   Max.   :2.0000   Max.   :14.90   Max.   :8.000
\end{verbatim}

\hypertarget{estaduxedsticas-descriptivas-y-correlaciones}{%
\subsection{Estadísticas descriptivas y
correlaciones}\label{estaduxedsticas-descriptivas-y-correlaciones}}

\begin{Shaded}
\begin{Highlighting}[]
\CommentTok{\# Matriz de correlación y dispersión}
\FunctionTok{ggcorr}\NormalTok{(vinos, }\AttributeTok{label =} \ConstantTok{TRUE}\NormalTok{)}
\end{Highlighting}
\end{Shaded}

\includegraphics{pareja2_files/figure-latex/unnamed-chunk-2-1.pdf}

\begin{Shaded}
\begin{Highlighting}[]
\FunctionTok{corrplot}\NormalTok{(}\FunctionTok{cor}\NormalTok{(vinos), }\AttributeTok{method =} \StringTok{\textquotesingle{}color\textquotesingle{}}\NormalTok{, }\AttributeTok{tl.cex =} \FloatTok{0.7}\NormalTok{)}
\end{Highlighting}
\end{Shaded}

\includegraphics{pareja2_files/figure-latex/unnamed-chunk-2-2.pdf}

\begin{Shaded}
\begin{Highlighting}[]
\FunctionTok{ggpairs}\NormalTok{(vinos, }\AttributeTok{columns =} \DecValTok{1}\SpecialCharTok{:}\DecValTok{12}\NormalTok{, }\AttributeTok{title =} \StringTok{"Matriz de correlaciones y dispersión"}\NormalTok{)}
\end{Highlighting}
\end{Shaded}

\includegraphics{pareja2_files/figure-latex/unnamed-chunk-2-3.pdf}

\begin{Shaded}
\begin{Highlighting}[]
\FunctionTok{cor}\NormalTok{(vinos}\SpecialCharTok{$}\NormalTok{quality, vinos[, }\DecValTok{1}\SpecialCharTok{:}\DecValTok{12}\NormalTok{])}
\end{Highlighting}
\end{Shaded}

\begin{verbatim}
##      fixed.acidity volatile.acidity citric.acid residual.sugar  chlorides
## [1,]     0.1240516       -0.3905578   0.2263725     0.01373164 -0.1289066
##      free.sulfur.dioxide total.sulfur.dioxide    density          pH sulphates
## [1,]         -0.05065606           -0.1851003 -0.1749192 -0.05773139 0.2513971
##        alcohol quality
## [1,] 0.4761663       1
\end{verbatim}

\hypertarget{divisiuxf3n-en-traintest}{%
\subsection{División en train/test}\label{divisiuxf3n-en-traintest}}

\begin{Shaded}
\begin{Highlighting}[]
\FunctionTok{set.seed}\NormalTok{(}\DecValTok{1}\NormalTok{)}
\NormalTok{train\_idx }\OtherTok{\textless{}{-}} \FunctionTok{sample}\NormalTok{(}\FunctionTok{seq\_len}\NormalTok{(}\FunctionTok{nrow}\NormalTok{(vinos)), }\AttributeTok{size =} \FloatTok{0.8} \SpecialCharTok{*} \FunctionTok{nrow}\NormalTok{(vinos))}
\NormalTok{train }\OtherTok{\textless{}{-}}\NormalTok{ vinos[train\_idx, ]}
\NormalTok{test }\OtherTok{\textless{}{-}}\NormalTok{ vinos[}\SpecialCharTok{{-}}\NormalTok{train\_idx, ]}
\NormalTok{X\_train }\OtherTok{\textless{}{-}} \FunctionTok{as.matrix}\NormalTok{(train[, }\SpecialCharTok{!}\NormalTok{(}\FunctionTok{names}\NormalTok{(train) }\SpecialCharTok{\%in\%} \FunctionTok{c}\NormalTok{(}\StringTok{\textquotesingle{}quality\textquotesingle{}}\NormalTok{))])}
\NormalTok{y\_train }\OtherTok{\textless{}{-}}\NormalTok{ train}\SpecialCharTok{$}\NormalTok{quality}
\NormalTok{X\_test }\OtherTok{\textless{}{-}} \FunctionTok{as.matrix}\NormalTok{(test[, }\SpecialCharTok{!}\NormalTok{(}\FunctionTok{names}\NormalTok{(test) }\SpecialCharTok{\%in\%} \FunctionTok{c}\NormalTok{(}\StringTok{\textquotesingle{}quality\textquotesingle{}}\NormalTok{))])}
\NormalTok{y\_test }\OtherTok{\textless{}{-}}\NormalTok{ test}\SpecialCharTok{$}\NormalTok{quality}
\end{Highlighting}
\end{Shaded}

\hypertarget{modelos-de-regresiuxf3n}{%
\section{Modelos de regresión}\label{modelos-de-regresiuxf3n}}

\hypertarget{regresiuxf3n-lineal-muxfaltiple}{%
\subsection{Regresión lineal
múltiple}\label{regresiuxf3n-lineal-muxfaltiple}}

\begin{Shaded}
\begin{Highlighting}[]
\NormalTok{lm\_fit }\OtherTok{\textless{}{-}} \FunctionTok{lm}\NormalTok{(quality }\SpecialCharTok{\textasciitilde{}}\NormalTok{ ., }\AttributeTok{data =}\NormalTok{ train)}
\FunctionTok{summary}\NormalTok{(lm\_fit)}
\end{Highlighting}
\end{Shaded}

\begin{verbatim}
## 
## Call:
## lm(formula = quality ~ ., data = train)
## 
## Residuals:
##      Min       1Q   Median       3Q      Max 
## -2.52213 -0.35627 -0.04738  0.44215  2.00517 
## 
## Coefficients:
##                        Estimate Std. Error t value Pr(>|t|)    
## (Intercept)           2.342e+01  2.323e+01   1.008 0.313441    
## fixed.acidity         2.725e-02  2.862e-02   0.952 0.341210    
## volatile.acidity     -1.032e+00  1.353e-01  -7.625 4.77e-14 ***
## citric.acid          -1.857e-01  1.626e-01  -1.142 0.253523    
## residual.sugar        2.223e-02  1.623e-02   1.370 0.170964    
## chlorides            -1.456e+00  4.849e-01  -3.002 0.002731 ** 
## free.sulfur.dioxide   4.483e-03  2.452e-03   1.828 0.067725 .  
## total.sulfur.dioxide -3.086e-03  8.191e-04  -3.767 0.000173 ***
## density              -1.976e+01  2.372e+01  -0.833 0.404986    
## pH                   -3.457e-01  2.165e-01  -1.597 0.110515    
## sulphates             9.192e-01  1.257e-01   7.311 4.69e-13 ***
## alcohol               2.834e-01  2.909e-02   9.741  < 2e-16 ***
## ---
## Signif. codes:  0 '***' 0.001 '**' 0.01 '*' 0.05 '.' 0.1 ' ' 1
## 
## Residual standard error: 0.6474 on 1267 degrees of freedom
## Multiple R-squared:  0.3536, Adjusted R-squared:  0.348 
## F-statistic: 63.02 on 11 and 1267 DF,  p-value: < 2.2e-16
\end{verbatim}

\begin{Shaded}
\begin{Highlighting}[]
\NormalTok{lm\_pred }\OtherTok{\textless{}{-}} \FunctionTok{predict}\NormalTok{(lm\_fit, }\AttributeTok{newdata =}\NormalTok{ test)}
\FunctionTok{cat}\NormalTok{(}\StringTok{\textquotesingle{}Regresión lineal múltiple:}\SpecialCharTok{\textbackslash{}n}\StringTok{\textquotesingle{}}\NormalTok{)}
\end{Highlighting}
\end{Shaded}

\begin{verbatim}
## Regresión lineal múltiple:
\end{verbatim}

\begin{Shaded}
\begin{Highlighting}[]
\FunctionTok{cat}\NormalTok{(}\StringTok{\textquotesingle{}Error Cuadrático Medio (MSE):\textquotesingle{}}\NormalTok{, }\FunctionTok{mean}\NormalTok{((test}\SpecialCharTok{$}\NormalTok{quality }\SpecialCharTok{{-}}\NormalTok{ lm\_pred)}\SpecialCharTok{\^{}}\DecValTok{2}\NormalTok{), }\StringTok{\textquotesingle{}}\SpecialCharTok{\textbackslash{}n}\StringTok{\textquotesingle{}}\NormalTok{)}
\end{Highlighting}
\end{Shaded}

\begin{verbatim}
## Error Cuadrático Medio (MSE): 0.4264839
\end{verbatim}

\begin{Shaded}
\begin{Highlighting}[]
\FunctionTok{cat}\NormalTok{(}\StringTok{\textquotesingle{}R2:\textquotesingle{}}\NormalTok{, }\FunctionTok{summary}\NormalTok{(lm\_fit)}\SpecialCharTok{$}\NormalTok{r.squared, }\StringTok{\textquotesingle{}}\SpecialCharTok{\textbackslash{}n}\StringTok{\textquotesingle{}}\NormalTok{)}
\end{Highlighting}
\end{Shaded}

\begin{verbatim}
## R2: 0.3536435
\end{verbatim}

\hypertarget{gruxe1ficos-de-residuos-lineal}{%
\subsubsection{Gráficos de residuos
(lineal)}\label{gruxe1ficos-de-residuos-lineal}}

\begin{Shaded}
\begin{Highlighting}[]
\NormalTok{residuos }\OtherTok{\textless{}{-}} \FunctionTok{resid}\NormalTok{(lm\_fit)}
\NormalTok{ajustados }\OtherTok{\textless{}{-}} \FunctionTok{fitted}\NormalTok{(lm\_fit)}
\FunctionTok{hist}\NormalTok{(residuos, }\AttributeTok{breaks =} \DecValTok{30}\NormalTok{, }\AttributeTok{col =} \StringTok{"steelblue"}\NormalTok{, }\AttributeTok{main =} \StringTok{"Distribución de residuos"}\NormalTok{, }\AttributeTok{xlab =} \StringTok{"Residuos"}\NormalTok{)}
\end{Highlighting}
\end{Shaded}

\includegraphics{pareja2_files/figure-latex/unnamed-chunk-5-1.pdf}

\begin{Shaded}
\begin{Highlighting}[]
\FunctionTok{plot}\NormalTok{(ajustados, residuos, }\AttributeTok{xlab =} \StringTok{"Valores ajustados"}\NormalTok{, }\AttributeTok{ylab =} \StringTok{"Residuos"}\NormalTok{, }\AttributeTok{main =} \StringTok{"Residuos vs Valores ajustados"}\NormalTok{, }\AttributeTok{pch =} \DecValTok{20}\NormalTok{, }\AttributeTok{col =} \StringTok{"darkred"}\NormalTok{)}
\FunctionTok{abline}\NormalTok{(}\AttributeTok{h =} \DecValTok{0}\NormalTok{, }\AttributeTok{lty =} \DecValTok{2}\NormalTok{, }\AttributeTok{col =} \StringTok{"gray"}\NormalTok{)}
\end{Highlighting}
\end{Shaded}

\includegraphics{pareja2_files/figure-latex/unnamed-chunk-5-2.pdf}

\hypertarget{ridge}{%
\subsection{Ridge}\label{ridge}}

\begin{Shaded}
\begin{Highlighting}[]
\NormalTok{cv\_ridge }\OtherTok{\textless{}{-}} \FunctionTok{cv.glmnet}\NormalTok{(X\_train, y\_train, }\AttributeTok{alpha =} \DecValTok{0}\NormalTok{)}
\NormalTok{best\_lambda\_ridge }\OtherTok{\textless{}{-}}\NormalTok{ cv\_ridge}\SpecialCharTok{$}\NormalTok{lambda.min}
\NormalTok{ridge\_pred }\OtherTok{\textless{}{-}} \FunctionTok{predict}\NormalTok{(cv\_ridge, }\AttributeTok{s =}\NormalTok{ best\_lambda\_ridge, }\AttributeTok{newx =}\NormalTok{ X\_test)}
\NormalTok{r2\_ridge }\OtherTok{\textless{}{-}} \DecValTok{1} \SpecialCharTok{{-}} \FunctionTok{sum}\NormalTok{((y\_test }\SpecialCharTok{{-}}\NormalTok{ ridge\_pred)}\SpecialCharTok{\^{}}\DecValTok{2}\NormalTok{) }\SpecialCharTok{/} \FunctionTok{sum}\NormalTok{((y\_test }\SpecialCharTok{{-}} \FunctionTok{mean}\NormalTok{(y\_test))}\SpecialCharTok{\^{}}\DecValTok{2}\NormalTok{)}
\FunctionTok{cat}\NormalTok{(}\StringTok{\textquotesingle{}}\SpecialCharTok{\textbackslash{}n}\StringTok{Ridge:}\SpecialCharTok{\textbackslash{}n}\StringTok{\textquotesingle{}}\NormalTok{)}
\end{Highlighting}
\end{Shaded}

\begin{verbatim}
## 
## Ridge:
\end{verbatim}

\begin{Shaded}
\begin{Highlighting}[]
\FunctionTok{cat}\NormalTok{(}\StringTok{"Error Cuadrático Medio (MSE): "}\NormalTok{, }\FunctionTok{mean}\NormalTok{((y\_test }\SpecialCharTok{{-}}\NormalTok{ ridge\_pred)}\SpecialCharTok{\^{}}\DecValTok{2}\NormalTok{), }\StringTok{\textquotesingle{}}\SpecialCharTok{\textbackslash{}n}\StringTok{\textquotesingle{}}\NormalTok{)}
\end{Highlighting}
\end{Shaded}

\begin{verbatim}
## Error Cuadrático Medio (MSE):  0.4277567
\end{verbatim}

\begin{Shaded}
\begin{Highlighting}[]
\FunctionTok{cat}\NormalTok{(}\StringTok{\textquotesingle{}R2:\textquotesingle{}}\NormalTok{, r2\_ridge, }\StringTok{\textquotesingle{}}\SpecialCharTok{\textbackslash{}n}\StringTok{\textquotesingle{}}\NormalTok{)}
\end{Highlighting}
\end{Shaded}

\begin{verbatim}
## R2: 0.3792105
\end{verbatim}

\hypertarget{gruxe1ficos-de-residuos-ridge}{%
\subsubsection{Gráficos de residuos
(Ridge)}\label{gruxe1ficos-de-residuos-ridge}}

\begin{Shaded}
\begin{Highlighting}[]
\NormalTok{residuos\_ridge }\OtherTok{\textless{}{-}}\NormalTok{ y\_test}\SpecialCharTok{{-}}\NormalTok{ridge\_pred}
\FunctionTok{hist}\NormalTok{(residuos\_ridge, }\AttributeTok{breaks =} \DecValTok{30}\NormalTok{, }\AttributeTok{col =} \StringTok{"tomato"}\NormalTok{, }\AttributeTok{main =} \StringTok{"Residuos del modelo penalizado"}\NormalTok{, }\AttributeTok{xlab =} \StringTok{"Residuos"}\NormalTok{)}
\end{Highlighting}
\end{Shaded}

\includegraphics{pareja2_files/figure-latex/unnamed-chunk-7-1.pdf}

\begin{Shaded}
\begin{Highlighting}[]
\FunctionTok{plot}\NormalTok{(ridge\_pred, residuos\_ridge, }\AttributeTok{xlab =} \StringTok{"Valores ajustados"}\NormalTok{, }\AttributeTok{ylab =} \StringTok{"Residuos"}\NormalTok{, }\AttributeTok{main =} \StringTok{"Residuos vs Valores ajustados (penalizado)"}\NormalTok{, }\AttributeTok{pch =} \DecValTok{20}\NormalTok{, }\AttributeTok{col =} \StringTok{"blue"}\NormalTok{)}
\FunctionTok{abline}\NormalTok{(}\AttributeTok{h =} \DecValTok{0}\NormalTok{, }\AttributeTok{lty =} \DecValTok{2}\NormalTok{, }\AttributeTok{col =} \StringTok{"gray"}\NormalTok{)}
\end{Highlighting}
\end{Shaded}

\includegraphics{pareja2_files/figure-latex/unnamed-chunk-7-2.pdf}

\hypertarget{lasso}{%
\subsection{LASSO}\label{lasso}}

\begin{Shaded}
\begin{Highlighting}[]
\NormalTok{cv\_lasso }\OtherTok{\textless{}{-}} \FunctionTok{cv.glmnet}\NormalTok{(X\_train, y\_train, }\AttributeTok{alpha =} \DecValTok{1}\NormalTok{)}
\NormalTok{best\_lambda\_lasso }\OtherTok{\textless{}{-}}\NormalTok{ cv\_lasso}\SpecialCharTok{$}\NormalTok{lambda.min}
\NormalTok{lasso\_pred }\OtherTok{\textless{}{-}} \FunctionTok{predict}\NormalTok{(cv\_lasso, }\AttributeTok{s =}\NormalTok{ best\_lambda\_lasso, }\AttributeTok{newx =}\NormalTok{ X\_test)}
\NormalTok{r2\_lasso }\OtherTok{\textless{}{-}} \DecValTok{1} \SpecialCharTok{{-}} \FunctionTok{sum}\NormalTok{((y\_test }\SpecialCharTok{{-}}\NormalTok{ lasso\_pred)}\SpecialCharTok{\^{}}\DecValTok{2}\NormalTok{) }\SpecialCharTok{/} \FunctionTok{sum}\NormalTok{((y\_test }\SpecialCharTok{{-}} \FunctionTok{mean}\NormalTok{(y\_test))}\SpecialCharTok{\^{}}\DecValTok{2}\NormalTok{)}
\FunctionTok{cat}\NormalTok{(}\StringTok{\textquotesingle{}}\SpecialCharTok{\textbackslash{}n}\StringTok{LASSO:}\SpecialCharTok{\textbackslash{}n}\StringTok{\textquotesingle{}}\NormalTok{)}
\end{Highlighting}
\end{Shaded}

\begin{verbatim}
## 
## LASSO:
\end{verbatim}

\begin{Shaded}
\begin{Highlighting}[]
\FunctionTok{cat}\NormalTok{(}\StringTok{"Error Cuadrático Medio (MSE): "}\NormalTok{, }\FunctionTok{mean}\NormalTok{((y\_test }\SpecialCharTok{{-}}\NormalTok{ lasso\_pred)}\SpecialCharTok{\^{}}\DecValTok{2}\NormalTok{), }\StringTok{\textquotesingle{}}\SpecialCharTok{\textbackslash{}n}\StringTok{\textquotesingle{}}\NormalTok{)}
\end{Highlighting}
\end{Shaded}

\begin{verbatim}
## Error Cuadrático Medio (MSE):  0.430363
\end{verbatim}

\begin{Shaded}
\begin{Highlighting}[]
\FunctionTok{cat}\NormalTok{(}\StringTok{\textquotesingle{}R2:\textquotesingle{}}\NormalTok{, r2\_lasso, }\StringTok{\textquotesingle{}}\SpecialCharTok{\textbackslash{}n}\StringTok{\textquotesingle{}}\NormalTok{)}
\end{Highlighting}
\end{Shaded}

\begin{verbatim}
## R2: 0.3754281
\end{verbatim}

\hypertarget{gruxe1ficos-de-residuos-lasso}{%
\subsubsection{Gráficos de residuos
(LASSO)}\label{gruxe1ficos-de-residuos-lasso}}

\begin{Shaded}
\begin{Highlighting}[]
\NormalTok{residuos\_lasso }\OtherTok{\textless{}{-}}\NormalTok{ y\_test}\SpecialCharTok{{-}}\NormalTok{lasso\_pred}
\FunctionTok{hist}\NormalTok{(residuos\_lasso, }\AttributeTok{breaks =} \DecValTok{30}\NormalTok{, }\AttributeTok{col =} \StringTok{"tomato"}\NormalTok{, }\AttributeTok{main =} \StringTok{"Residuos del modelo penalizado"}\NormalTok{, }\AttributeTok{xlab =} \StringTok{"Residuos"}\NormalTok{)}
\end{Highlighting}
\end{Shaded}

\includegraphics{pareja2_files/figure-latex/unnamed-chunk-9-1.pdf}

\begin{Shaded}
\begin{Highlighting}[]
\FunctionTok{plot}\NormalTok{(lasso\_pred, residuos\_lasso, }\AttributeTok{xlab =} \StringTok{"Valores ajustados"}\NormalTok{, }\AttributeTok{ylab =} \StringTok{"Residuos"}\NormalTok{, }\AttributeTok{main =} \StringTok{"Residuos vs Valores ajustados (penalizado)"}\NormalTok{, }\AttributeTok{pch =} \DecValTok{20}\NormalTok{, }\AttributeTok{col =} \StringTok{"blue"}\NormalTok{)}
\FunctionTok{abline}\NormalTok{(}\AttributeTok{h =} \DecValTok{0}\NormalTok{, }\AttributeTok{lty =} \DecValTok{2}\NormalTok{, }\AttributeTok{col =} \StringTok{"gray"}\NormalTok{)}
\end{Highlighting}
\end{Shaded}

\includegraphics{pareja2_files/figure-latex/unnamed-chunk-9-2.pdf}

\end{document}
